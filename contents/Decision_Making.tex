\section{Decision Making}

\subsection{Decision Tree construction}
The essence of the problem can be viewed as a decision-making task — specifically, choosing a recipe from a recipe book. To model this process systematically, a decision tree is constructed to represent the possible choices and outcomes. In this case, some problems would be:
\begin{itemize}
    \item What are the possible outcomes after recommending a recipe?
    \item How to find the probability of these outcomes?
    \item How to find the utility of these outcomes?
\end{itemize}
Our design is simple. The recipe will either be feasible or not, given the foods in the fridge. The feasibility of the recipe is assessed using a Bayesian network, and the utility of a feasible recipe is assessed using a supervised machine learning algorithm. The utility of an infeasible recipe is set to 0. A diagram illustration is given in appendix as Figure \ref{fig:DN}.
\subsection{Decision Making Process}
As a rational agent, our bot aims to choose the recipe that maximizes the expected utility. Here is the meaning of the notations we used in our formula:
\begin{itemize}
    \item Rec: The final recommendation 
    \item R: The recipe book (set of all recipes)
    \item r: A specific recipe
    \item s: A binary value, representing feasible or not
    \item P: The feasibility function
    \item U: The utility function
\end{itemize}
With these notations, our algorithm can be represented as:
$$
\text{Rec} = \arg\max_{r \in R} \mathbb{E}[U \mid r] = \arg\max_{r \in R} \sum_s P(s \mid r) \, U(s, r)
$$
Since U(fail, r) = 0 by design, the formula can be simplified to:
$$
\text{Rec} = \arg\max_{r \in R} P(feasible \mid r) \, U(feasible, r)
$$