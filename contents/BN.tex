\section{Bayesian Network Inferencing}
The \textbf{feasibility} of preparing recipe $\recipe_i$ depends on state of foods available in the \textbf{fridge} that is in the requirement set $\requirements_{\recipe_i}$. We assume that the state of each food is dependent on three characteristics of food $\food_i$: \textbf{food type} $\fty_{\food_i}$, \textbf{storage type} $\sty_{\food_i}$, and \textbf{days in the fridge} $\days_{\food_i}$. 

Due to the partial observability of the true state of food items in the fridge, \textbf{\underline{Bayesian Network}} is used for handling the uncertainty and infer the probability of the successful preparation of a specific recipe. The \underline{\textbf{pgmpy}} (\cite{Ankan2024}) library is used to construct \textbf{\underline{Bayesian Network}} and perform the inference task.

\subsection{BN structure}
The visual representation of the overall structure can be seen from the picture Figure~\ref{fig:BN}

The feasibility of the recipe depends on feasibility of the food items, in other words, whether the food has expired. To model this relationship, we first construct a sub-Bayesian Network for each of the food item. In the sub-network, the node $expired_{\food_i}$, which is a random variable denoting the event whether the food is expired, has three parents nodes which representing the random variables for food type $\fty_{\food_i}$, storage type $\sty_{\food_i}$, and days in the fridge $\days_{\food_i}$ respectively. The parent nodes representing the observed characteristics of the food.

The probability of successfully preparing the recipe is denoting by random variable $feasibility_{\recipe_i}$. We assume that the feasibility of each food item is conditionally independent of each other. As such, the node $feasibility_{\recipe_i}$ will have multiple parents that denoting the random variable of $expired_{\food_i}$ for each $\food_i$ in recipe's requirements respectively.

Hence, we can derive following formulae for the probability of successfully preparing a recipe, i.e. feasibility of the recipe (Assume all food items required by the recipe can be found in the fridge):

\begin{align*}
&P(feasibility_{\recipe_i} | \Fridge) \\
&= P(feasibility_{\recipe_i} | expired_{\food_1}, \dots expired_{\food_{n}})\prod_{k=1}^nP(expired_{\food_i} | \food_{i}) \\
&=P(feasibility_{\recipe_i}  | expired_{\food_1}, \dots expired_{\food_{n}})\prod_{k=1}^nP(expired_{\food_i} | \fty_{\food_i}, \sty_{\food_i}, \days_{\food_i}) \\
&\Fridge \text{ denotes the current state of the fridge with all the food available } \\
&\food_{1} \dots \food_{n} \in \Fridge
\end{align*}

\subsection{Inference Framework}
Here is the process of building the Bayesian Network and perform the inference process with the help of pgmpy.

\begin{itemize}[topsep=0pt, partopsep=0pt, itemsep=2pt, parsep=0pt]
    \item Extract food items from the fridge that is required by a recipe
    \item Build $BN_{\food_i}$ for each food extracted
    \item Connect $BN_{\food_i}$ to node $feasibility_{\recipe_i}$
    \item Extract the evidence from each food items
    \item Run variable elimination inference algorithm
\end{itemize}

The code that generates this part is presented in \ref{app:algorithms}

\subsection{Maximum Likelihood Estimate}
We estimate the Conditional Probability Table of each node using a maximum likelihood estimate. We would estimate $P(a)=\frac{\#(a)}{N}$ where $\#(a)$ denotes the count of occurrences of $a$ in the dataset and $N$ is the number of samples. We can estimate conditional probability in a similar way: $P(a|b)=\frac{\#(a\wedge b)}{\#(b)}$. The complete set of data is in \ref{app:data}